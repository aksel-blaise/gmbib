\listfiles
\documentclass[review]{elsarticle}

\usepackage{lineno,hyperref}
\modulolinenumbers[5]

\journal{Advances in Archaeological Practice}

%%%%%%%%%%%%%%%%%%%%%%%
%% Elsevier bibliography styles
%%%%%%%%%%%%%%%%%%%%%%%
%% To change the style, put a % in front of the second line of the current style and
%% remove the % from the second line of the style you would like to use.
%%%%%%%%%%%%%%%%%%%%%%%

% Numbered
% \bibliographystyle{model1-num-names}

%% Numbered without titles
% \bibliographystyle{model1a-num-names}

%% Harvard
% \bibliographystyle{model2-names}\biboptions{authoryear}

%% Vancouver numbered
% \usepackage{numcompress}\bibliographystyle{model3-num-names}

%% Vancouver name/year
% \usepackage{numcompress}\bibliographystyle{model4-names}\biboptions{authoryear}

%% APA style
% \bibliographystyle{model5-names}\biboptions{authoryear}

%% AMA style
% \usepackage{numcompress}\bibliographystyle{model6-num-names}

%% `Elsevier LaTeX' style, distributed in TeX Live 2019
\bibliographystyle{elsarticle-num}
% \usepackage{numcompress}\bibliographystyle{elsarticle-num-names}
% \bibliographystyle{elsarticle-harv}\biboptions{authoryear}
%%%%%%%%%%%%%%%%%%%%%%%

\begin{document}

\begin{frontmatter}

\title{A Bibliometric Study of Geometric Morphometrics in Archaeology}

%% Group authors per affiliation:
\author{Robert Z. Selden, Jr.\textsuperscript{a,b,c*}, Christian S. Hoggard\textsuperscript{d,e}, and Sarah Y. Stark\textsuperscript{}}
\address[1]{Heritage Research Center, Stephen F. Austin State University, US}
\address[2]{Cultural Heritage Department, Jean Monnet University, FR}
\address[3]{ORCID ID \href{http://orcid.org/0000-0002-1789-8449}{0000-0002-1789-8449}}
\address[4]{Department of Archaeology and Anthropology, University of Southampton, UK}
\address[5]{ORCID ID \href{http://orcid.org/0000-0002-0022-3605}{0000-0002-0022-3605}}
\cortext[cor1]{Corresponding author, Robert Z. Selden, Jr. (zselden@sfasu.edu)}

\begin{abstract}
Geometric morphometric publications and their cited references were harvested from Scopus, and used in an exploratory bibliometric analysis. The citation network was subsequently filtered to include only those nodes with a degree of two or higher. Network statistics were calculated for in-degree and modularity. Results of the modularity analysis indicate seven sub-communities within the network, based upon common citation practices. In-degree and out-degree were used to identify and illustrate publications and references central to each sub-community. Using the citation network as an epistemological tool, practitioners can identify schools of thought or practice, references with the highest overall authority, references central to each school of thought or practice, the within-domain publications cited most, explore the progression of those publications, and interact with the cited works in graphical form.
\end{abstract}

\begin{keyword}
bibliometrics, geometric morphometrics, citation network, network analysis, literature review
\end{keyword}

\end{frontmatter}

\linenumbers

\section*{}



\bibliography{mybibfile}

\end{document}